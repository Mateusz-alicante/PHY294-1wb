\section{Introduction}

This lab will analyze the photoelectric effect by measuring the stopping potential and photocurrent for different wavelengths of light. The photoelectric effect is the emission of electrons from a material when light is shone on it. The energy of the electrons emitted is proportional to the frequency of the light, and the stopping potential is the voltage required to stop the electrons from reaching the anode. The stopping potential is proportional to the frequency of the light, and the proportionality constant is Planck's constant. The photocurrent is the current of electrons emitted from the material, and it is proportional to the intensity of the light.

The photoelectric effect was one of the first pieces of evidence for the quantization of energy. The photoelectric effect is important in many applications, such as photovoltaic cells.

\subsection{Historical background:}

JJ Thompson was the first to discover the photoelectric effect. He hypothesized that light was absorbed by the electrons in a continuous manner, and when a certain amount of
energy was reached, the electrons were released.

The German physicist Philip Lenard performed a key experiment in which he showed that the current was proportional to the intensity of the incident light, as well as that the photocurrent
only appeared after a certain threshold frequency was reached. He also found that if a potential was applied in the opposite direction of the emitted electrons, the current would decrease until it reached zero (Stopping potential).

Another milestone was reached by Albert Einstein in 1905, when he proposed that light was quantized in packets of energy called photons. This explained the threshold frequency and the stopping potential found by Lenard.

\subsection{Theoretical background:}

Einsteins key finding was that the energy of the electron is given by the equation:

\begin{equation}
    \label{eq:energy}
    E = hf
\end{equation}

where $E$ is the energy of the electron, $h$ is Planck's constant, $f$ is the frequency of the light.

This knowladge, combined with the findings about the photoelectric effect that the kinetic energy of the ejected photon is the difference between the energy of the photon and the work function of the material, as well as the stopping potential, we can derive the following equation for the stopping potential:

\begin{equation}
    \label{eq:stopping}
    V_{s} = \frac{h}{e}(f-f_0)
\end{equation}

where $V_s$ is the stopping potential, $h$ is Planck's constant, $e$ is the charge of an electron, $f$ is the frequency of the light, and $f_0$ is the threshold frequency.


\subsection{Experiment:}

In the first part of the experiment we will obtain Planck's constant, the work function for the phototube, as well as the cutout frequency for the phototube. We will do this by measuring the stopping potential for different wavelengths of light.

The second part consists in measuring the stopping voltage and photocurrent as a function of incident light intensity. This will help us gain insight into the differences between the original and Einstein's model of the photoelectric effect.

In the third part we measure the time delay between the light being turned on and the current reaching a steady state. This will help us understand the time it takes for the electrons to be emitted from the material to draw conclusions about different explanations of the photoelectric effect.