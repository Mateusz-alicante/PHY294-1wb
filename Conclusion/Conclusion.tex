\section{Conclusion}

In the first part of the lab we discussed how the stopping voltage varies with incident wavelength, and used this to calculate key quantities: Planck's
constant, the work function and the cutout frequency for the phototube.

In the second part we look at the variations in stopping voltage and photocurrent with incident light intensity.
This helped us understand the differences between the (continuous) original and Einstein's (quantized) model of the photoelectric effect.

In the third part we measured the time delay between the light being turned on and the current reaching a steady state.
This helped us understand the time it takes for the electrons to be emitted from the material,
and conclude that the original model of the photoelectric effect is not sufficient to explain the data, while
Einstein's model is more complete by considering the quantization of light.